\documentclass{article}
\usepackage{blindtext}
\usepackage[a4paper, total={6in, 8in}]{geometry}
\usepackage{amsfonts}
\usepackage{amsmath,amsfonts, amssymb, epsfig, amscd, graphicx, latexsym, latexsym, amsthm,tikz, mdframed, float,enumitem, comment}
\usepackage{multicol}
\usepackage{indentfirst}

\title{Teaching Statement}
\author{Fırtına Küçük}
\date{}



\begin{document}
\maketitle

\newtheorem{lemma}{Lemma}[section]
\newtheorem{theorem}[lemma]{Theorem}
\newtheorem{prop}[lemma]{Proposition}
\newtheorem{cor}[lemma]{Corollary}
\newtheorem{conj}[lemma]{Conjecture}
\newtheorem{claim}[lemma]{Claim}
\newtheorem{definition}[lemma]{Definition}
\newtheorem{remark}[lemma]{Remark}


\newcommand{\mylabel}[2]{#2\def\@currentlabel{#2}\label{#1}}






\newcommand{\idele}{id{\'e}le}

\newcommand{\ccS}{\mathcal{S}}
\newcommand{\ccF}{\mathcal{F}}
\newcommand{\ccA}{\mathcal{A}}
\newcommand{\ccG}{\mathcal{G}}
\newcommand{\ccP}{\mathcal{P}}
\newcommand{\ccB}{\mathcal{B}}
\newcommand{\ccC}{\mathcal{C}}
\newcommand{\ccD}{\mathcal{D}}
\newcommand{\ccE}{\mathcal{E}}
\newcommand{\ccK}{\mathcal{K}}
\newcommand{\ccU}{\mathcal{U}}
\newcommand{\ccV}{\mathcal{V}}
\newcommand{\ccO}{\mathcal{O}}
\newcommand{\ccL}{\mathcal{L}}
\newcommand{\ccW}{\mathcal{W}}
\newcommand{\ccN}{\mathcal{N}}
\newcommand{\ccH}{\mathcal{H}}



\newcommand{\bbA}{{\mathbb A}}
\newcommand{\bbG}{{\mathbb G}}
\newcommand{\bbH}{{\mathbb H}}
\newcommand{\bbP}{{\mathbb P}}
\newcommand{\bbC}{{\mathbb C}}
\newcommand{\bbF}{{\mathbb F}}
\newcommand{\bbK}{{\mathbb K}}
\newcommand{\bbL}{{\mathbb L}}
\newcommand{\bbQ}{{\mathbb Q}}
\newcommand{\bbR}{{\mathbb R}}
\newcommand{\bbZ}{{\mathbb Z}}
\newcommand{\bbN}{{\mathbb N}}
\newcommand{\bbT}{{\mathbb T}}
\newcommand{\bbX}{{\mathbb X}}

\newcommand{\bbAt}{{\mathbb A}^{\times}}
\newcommand{\bbGt}{{\mathbb G}^{\times}}
\newcommand{\bbHt}{{\mathbb H}^{\times}}
\newcommand{\bbPt}{{\mathbb P}^{\times}}
\newcommand{\bbCt}{{\mathbb C}^{\times}}
\newcommand{\bbFt}{{\mathbb F}^{\times}}
\newcommand{\bbKt}{{\mathbb K}^{\times}}
\newcommand{\bbLt}{{\mathbb L}^{\times}}
\newcommand{\bbQt}{{\mathbb Q}^{\times}}
\newcommand{\bbRt}{{\mathbb R}^{\times}}
\newcommand{\bbZt}{{\mathbb Z}^{\times}}
\newcommand{\bbNt}{{\mathbb N}^{\times}}
\newcommand{\bbTt}{{\mathbb T}^{\times}}

\newcommand{\fraa}{{\mathfrak a}}
\newcommand{\frab}{{\mathfrak b}}
\newcommand{\frakc}{{\mathfrak c}}
\newcommand{\frad}{{\mathfrak d}}
\newcommand{\frae}{{\mathfrak e}}
\newcommand{\fraf}{{\mathfrak f}}
\newcommand{\frag}{{\mathfrak g}}
\newcommand{\frah}{{\mathfrak h}}
\newcommand{\frai}{{\mathfrak i}}
\newcommand{\fraj}{{\mathfrak j}}
\newcommand{\frk}{{\mathfrak k}}
\newcommand{\fral}{{\mathfrak l}}
\newcommand{\fram}{{\mathfrak m}}
\newcommand{\fran}{{\mathfrak n}}
\newcommand{\frao}{{\mathfrak o}}
\newcommand{\frap}{{\mathfrak p}}
\newcommand{\fraq}{{\mathfrak q}}
\newcommand{\frar}{{\mathfrak r}}
\newcommand{\fras}{{\mathfrak s}}
\newcommand{\frat}{{\mathfrak t}}
\newcommand{\frau}{{\mathfrak u}}
\newcommand{\frav}{{\mathfrak v}}
\newcommand{\fraw}{{\mathfrak w}}
\newcommand{\frax}{{\mathfrak x}}
\newcommand{\fray}{{\mathfrak y}}
\newcommand{\fraz}{{\mathfrak z}}

\newcommand{\frakA}{{\mathfrak A}}
\newcommand{\frakB}{{\mathfrak B}}
\newcommand{\frakC}{{\mathfrak C}}
\newcommand{\frakD}{{\mathfrak D}}
\newcommand{\frakE}{{\mathfrak E}}
\newcommand{\frakF}{{\mathfrak F}}
\newcommand{\frakG}{{\mathfrak G}}
\newcommand{\frakH}{{\mathfrak H}}
\newcommand{\frakI}{{\mathfrak I}}
\newcommand{\frakJ}{{\mathfrak J}}
\newcommand{\frakK}{{\mathfrak K}}
\newcommand{\frakL}{{\mathfrak L}}
\newcommand{\frakM}{{\mathfrak M}}
\newcommand{\frakN}{{\mathfrak N}}
\newcommand{\frakO}{{\mathfrak O}}
\newcommand{\frakP}{{\mathfrak P}}
\newcommand{\frakQ}{{\mathfrak Q}}
\newcommand{\frakR}{{\mathfrak R}}
\newcommand{\frakS}{{\mathfrak S}}
\newcommand{\frakT}{{\mathfrak T}}
\newcommand{\frakU}{{\mathfrak U}}
\newcommand{\frakV}{{\mathfrak V}}
\newcommand{\frakW}{{\mathfrak W}}
\newcommand{\frakX}{{\mathfrak X}}
\newcommand{\frakY}{{\mathfrak Y}}
\newcommand{\frakZ}{{\mathfrak Z}}





\newcommand{\Zbar}{{\overline{\bbZ}}}
\newcommand{\kbar}{{\overline{k}}}
\newcommand{\Kbar}{{\overline{K}}}
\newcommand{\Fbar}{{\overline{\bbF}}}
\newcommand{\Vbar}{{\overline{V}}}
\newcommand{\Ybar}{{\overline{Y}}}
\newcommand{\Ubar}{{\overline{U}}}
\newcommand{\Cbar}{{\overline{C}}}
\newcommand{\Lbar}{{\overline{L}}}
\newcommand{\pbar}{{\overline{p}}}
\newcommand{\barQ}{{\overline{\bbQ}}}


\newcommand{\AK}{{\bbA^{\times}_K}}
\newcommand{\AF}{{\bbA^{\times}_F}}
\newcommand{\AM}{{\bbA^{\times}_M}}





\newcommand{\Zhat}{{\hat{\bbZ}}}


\newcommand{\OK}{{\mathcal{O}_K}}
\newcommand{\Ov}{{\mathcal{O}_v}}
\newcommand{\Tf}{{\mathbb{T}_f}}
\newcommand{\Tfac}{{\mathbb{T}_f^{ac}}}
\newcommand{\Tfxac}{{\mathbb{T}_{f,\chi}^{ac}}}
\newcommand{\Tfx}{{\mathbb{T}_{f,\chi}}}
\newcommand{\frakfx}{{\mathfrak f}_{\chi}}

\section*{My Teaching Philosophy}

\par It is certain that one of the most important factors which can either inspire or discourage a person from pursuing a career in a field is the teacher or lecturer. I have experienced the former as I was inspired to become a mathematician thanks to mainly my math olympiad teacher whom I met in the 8th grade; on the other hand, I have also seen a lot of people discouraged from being a mathematician or a scientist because of the toxic environment in the classrooms. One of the reasons that I am pursuing an academic career is to be an inspiration to people from younger generations, and encourage more people to become mathematicians or at least have fun while learning mathematics and appreciate its beauty. As a Ph.D. student who has been holding tutorials and problem sessions for almost six years, this has been my primary goal in teaching.

\vspace{5mm}

\par Creating a positive environment in which the students are unafraid to ask questions, even if they may seem like ``wrong'' questions or have ``obvious'' looking answers, is an essential ingredient to creating a non-toxic environment. In my tutorials and problem sessions, I regularly remind them to not hesitate to ask questions, either at the moment or in private after the tutorials. I try to make them be sure that when they ask questions, they will be answered without any judgement. On the other hand, asking the right questions is also very important in mathematics, and one should encourage their students to do this through positive reinforcement.
\vspace{5mm}

\par This does not only apply to questions but also to answers given by students. Even the experts in mathematics are making a lot of mistakes, so we should convince the students that making mistakes is an integral part of the learning process. It is also worthwhile to note the importance of encouraging students to notice the lecturer's mistakes to give feedback about it. As a student, getting a positive response during such situations increased both my self-esteem and my respect for the instructor. Now as a tutor and prospective lecturer, I thank my students for noticing my mistakes or providing an alternative better solution to the problem.

\vspace{5mm}

\par Providing a non-toxic environment to the students is one necessary part of the solution; however it is not sufficient. One should also inspire the students for continuing their careers in the field, or at least make them enjoy the lecture material and maintain their interest in the topic. Throughout my experience both as a student and a tutor, I have realized that there are several key steps for achieving that. 

\vspace{5mm}

\par The reason I liked mathematics is that unlike how it has been perceived by most students, it is not just about applying formulas and calculations, but also about understanding the reasoning behind them  and the relationships between different concepts. Mathematics is very undogmatic by its nature; if something is a theorem, then it should follow by applying the rules of logic either from definitions and axioms, or from other theorems. Therefore, I would be very happy if my prospective students could also appreciate this as I did. However, one should also realize that technical proofs can also sometimes be very scary, especially for students of mathematics modules from different majors. And in that case, their point of view can change from memorizing the formulas to memorizing proofs. I believe the secret to the solution to this problem lies in establishing the balance between rigor and intuition.

\vspace{5mm}

\par In mathematical research, rigor and formality are extremely important. Indeed, an ambiguity in the argument of a proof or a tiny mistake in details could lead to wrong results. However, focusing on rigor too much and not giving enough priority to intuition would lead the students not appreciate the material, internalize the concepts, and understand why the arguments are the way they are. Therefore, the lecturer's priority should be giving the students good intuition and motivation. After the students have a good ``picture'' in their mind of what is going on, the importance of rigor and formal proof could be much easier for them to appreciate. As Ravi Vakil said in his online lectures on algebraic geometry - AGITTOC, the important question is ``why is a thing'' instead of ``what is a thing''. Learning the proof should be a tool for answering this question, it should not be the end goal.

\vspace{5mm}

\par To give intuition to the students, choosing good examples and counter-examples is essential. Indeed, applying the arguments in the proof of a statement to some non-trivial examples, the students might already guess what the general argument should be. This does not only apply to the proof, but also to a general method for the solution of a problem, and more abstract concepts. On the other hand, counter-examples are helpful for students to understand the importance of details or common mistakes. For instance, if I notice a common mistake done in previous years, I warn them in my tutorials about it and show them why it leads to a wrong result by choosing a counter-example. 

\vspace{5mm}

\par In addition, starting the semester with some motivation about the topic in a broader sense will also get the students' attention to the lecture material throughout the semester. This includes why the subject is invented, why it is important in terms of its applications to other areas, or even examples from future content. This will also allow the student to realize that learning is not always linear. I firmly believe that the syllabus of the lecture should be self-contained as much as possible. However, although providing examples or applications from the future material at first might make the students get a bit out of their comfort zone, it will help them to internalize the material much better when the time comes. I also tend to do the opposite of this, by reminding them a bit of the previous content if I need to use it in my arguments. Additionally, I start my tutorials by telling them what we have recently covered and how we will progress this week in case the contents of the last two lectures are relevant.

\vspace{5mm}

\par I have also observed as a student that how a lecturer writes their lecture notes on the blackboard affect the student's learning. In my opinion, the notes should be as organized as possible; the closer it is to a well-structured math book (or an online lecture note) the better it would be. It is even better if it includes drawings and diagrams to improve the students' intuition. This method also gives some advantages to the people who could not attend the previous lectures. However this definitely should not mean that the lecture must consist purely of writing a book on the blackboard, as this would make the students get lost, and lose interest to the module in general. The lecture should include in-class discussions about the topic, and interaction with students, and so on. The notes should be written clearly, and be visible from the backseats in the classroom. Moreover using beamer slides, animations (like the ones in 3B1B youtube channel), and programming simulations could boost the understanding of the student and thus make the lecture much more interesting.

\vspace{5mm}

\par Finally, I should note that as a tutor, knowing and learning more in the field of abstract mathematics have affected my teaching simpler topics to my students in a much better way. I first realized it while I was a teaching assistant in Multivariable Calculus. Prior to that year, I had a chance to audit the Differential Geometry module and learn about tangent spaces and what a differential was in a much more abstract way than we did in Multivariable Calculus. I felt that I was much more confident with the material, and thanks to the knowledge I had, I had been able to provide a much better intuition about these concepts, for example, the multivariable chain rule. In the Introduction to Mathematics module that I have been tutoring for the last four years, I integrate my intuition from abstract algebra to explain to the students how to solve basic first-order equations.

\section*{My Teaching Experience and Future Plans}

\par I have been a teaching assistant and a tutor since 2017 - the last semester of my undergraduate at Koç University, and currently at University College Dublin (UCD), I have had a considerable amount of experience with different groups of students. The problem sessions that I conducted at Koç University were mainly aimed towards mathematics students; which were Real Analysis I, Complex Analysis, Introduction to Abstract Mathematics, Discrete Mathematics in addition to Differential Equations, and Multivariable Calculus which were aimed towards engineering and science students.

\vspace{5mm}

\par At UCD, my tutorials were aimed at non-mathematicians, namely Mathematics for Agriculture II, Mathematics: an Introduction, Intro to Analysis for Economics \& Finance, and Foundations of Maths for Computer Science I. 

\vspace{5mm}

\par Along my compulsory teaching duties, in 2017, I tutored Abstract Algebra II (Galois Theory) in Koç Office of Learning and Teaching (KOLT), and I was a section leader at CS-Bridge which took place at Koç University. CS-Bridge was a programme which was co-organized by Stanford University and Koç University, aimed at high school students that were interested in programming with no prior knowledge. Lecturers from Stanford University taught the students the essentials of programming via using Stanford's ACM library in JAVA.

\vspace{5mm}

\par In the following years, if I have the chance, I would like to organize modules on various topics aimed both at undergraduate students and graduate students. Most of them would be closely related to Algebraic Number Theory and Arithmetic Geometry, but also include topics as Homological Algebra, Category Theory, etc. It is my personal belief that exams are not the best way to measure a student's performance and knowledge, hence in the modules I would organize, I would like to focus more on homeworks, group projects, in-class presentations, etc.

\vspace{5mm}

\par I am also planning to give lectures related to my research in Nesin Mathematics Village in Şirince, a village in İzmir, Turkey, after I complete my Ph.D. dissertation. It is a mathematics village, founded by Ali Nesin who is a mathematician (and the son of the Turkish author Aziz Nesin) who dedicated himself to make people ranging from elementary school students to graduate students love mathematics by showing them the real essence of mathematics, teaching it the way it is. Many of my mathematician friends realized they loved mathematics in that village. Therefore it would be an incredible opportunity for me to get closer to my goal on inspiring and inviting people to see the beauty of mathematics, and the area that I am working on.


\end{document}
